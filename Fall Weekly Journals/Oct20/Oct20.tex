\documentclass[a4paper,10pt]{article}
\usepackage[utf8]{inputenc}

% increase margins
\usepackage{fullpage}
\usepackage[left=1in,top=1in,right=1in,bottom=1in,headheight=3ex,headsep=3ex]{geometry}

% this puts two lines between paragraphs and no indent
\usepackage[parfill]{parskip}

% set up colors
\usepackage{array, xcolor}
\usepackage{color,hyperref}
\usepackage{tabularx}
\definecolor{torontoblue}{HTML}{00204E}
\definecolor{linkblue}{HTML}{0000FF}

% define hyperlink style
\hypersetup{colorlinks,breaklinks,
            linkcolor=linkblue,urlcolor=linkblue,
            anchorcolor=linkblue,citecolor=linkblue}


%opening
\title{Weekly Journal}
\author{Leila Uy}



\begin{document}

\maketitle

%this is commented out, no need for abstract in  your weekly assignment
% \begin{abstract}
% 
% \end{abstract}

\section{Work Update}
For the last, two weeks I have been focusing on making progress on the Happiness Index and Clustering Code projects. I took a few days rest during the break and now I feel extremely energized to start back up again (so this is an apology for all the messages and emails I have/will send)!

\subsection{Happiness Index}
I read the two articles you sent my way and I know we are going to talk about preparing visualizations of the data on Thursday so I won't talk too much. 

The article uses 22 degrees C ($~$ 77 degrees F) to establish temperate climates and an index can be used with four absolute deviations from that point of reference. You suggested using WorldClim data for these maps so I was looking at the variables and it seems entirely possible since we have min., max., and average temperature. So, I will be experimenting with those variables tonight (on a low resolution for now)! 

\subsection{Clustering Code}
I sent out a When2Meet to Ryan, Soham and Jishnu to talk about:
\begin{itemize}
    \item Important data to include in our clustering code to identify regions where permanant tree crops can potentially grow in the future
    \item The reasonability of finding that data for our intended locations (a flashback to the soil data D:).
    \item Any potential roadblocks with clustering those data sets (different resolutions, etc.).
\end{itemize}
It's midterm season, so everyone is busy. Our best day is Thursday at 4-5pm if you want to join.

\section{Literature Review}


\begin{center}
\begin{table}[!ht]
\caption{The results found in the article "Climates Create Cultures"}\label{tab:1}
\begin{tabularx}{1\textwidth} { 
    | >{\centering\arraybackslash}X 
    | >{\centering\arraybackslash}X 
    | >{\centering\arraybackslash}X | }
    \hline
    \textbf{Poorer societies - Demanding climates} & \textbf{Richer societies - Demanding climates} & \textbf{Temperate climates} \\
    \hline
    Lower degrees of happiness, stronger motivation to improve happiness of others 
    & Higher degrees of happiness, weaker motivation to improve the happiness of others 
    & N/A \\
    \hline
    Voluntary workers have either predominantly egoistic or altruistic motives 
    & Voluntary workers tend to integrate egoistic and altruistic motives
    & Egoistic and altruistic motives for volunteer work did neither exclude nor include each other. \\
    \hline
    Working for money strongest & Working for fun strongest & Moderate for fun and money \\
    \hline
    Looking after one’s own interests first. Egoism is encouraged $>$ Cooperativeness. 
    & Acting unselfish and prosocial. Cooperativeness is encouraged $>$ Egoism. 
    & Concerns for self and other. Not relatively egoistic or relatively cooperative. \\
    \hline
    Autocratic leadership is seen as more effective 
    & Democratic leadership as more effective 
    & Neither downright autocratic nor downright democratic \\
    \hline
    \end{tabularx}
\end{table}
\end{center}

The main get-away from the two articles you provided is that lower-income countries with more demanding climates emphasize survival while higher-income countries with more demanding climates emphasize self-expression. Temperate locations are moderate and don't show to lean on either side. \cite{van2007climates,van2012climate}

I also read older articles by Evert Van de Vliert \cite{van2004colder} and one of the main things that caught my eye was the happiness index he used by Diener so, I'll take a deeper look into this \cite{diener2009factors}. 


% this info creates the bibliography
% YOU WILL NEED TO CHANGE THIS PATH TO THE LOCATION OF THE BIB file
\bibliography{./agclimate.bib}
\bibliographystyle{plain}


\end{document}
