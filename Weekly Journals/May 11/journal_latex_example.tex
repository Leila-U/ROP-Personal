\documentclass[a4paper,10pt]{article}
\usepackage[utf8]{inputenc}

% increase margins
\usepackage{fullpage}
\usepackage[left=1in,top=1in,right=1in,bottom=1in,headheight=3ex,headsep=3ex]{geometry}

% this puts two lines between paragraphs and no indent
\usepackage[parfill]{parskip}

% set up colors
\usepackage{array, xcolor}
\usepackage{color,hyperref}
\definecolor{torontoblue}{HTML}{00204E}
\definecolor{linkblue}{HTML}{0000FF}

% define hyperlink style
\hypersetup{colorlinks,breaklinks,
            linkcolor=linkblue,urlcolor=linkblue,
            anchorcolor=linkblue,citecolor=linkblue}


%opening
\title{Weekly Journal}
\author{Leila Uy}



\begin{document}

\maketitle

%this is commented out, no need for abstract in  your weekly assignment
% \begin{abstract}
% 
% \end{abstract}

\section{Work Update}

This past week I read the assigned article but spent most of my time focused on learning the concepts and architectures of 
parallel programming and the integration of parallel programming in R and GRASS.

\subsection{Parallel programming}
I watched many videos and readings about parallel programming because I was finding it difficult to understand academic
writings about it without knowing the basics. So, I started from the beginning. There are two major types of parallel 
architectures: distributed memory and shared memory. Distributed memory uses a message passing system, while shared memory 
access the main memory via uniform memory access (UMA) or non-uniform memory access. 

When parallel programming, there are different computation bounds aside from the CPU. Therefore although in theory adding
processors would linearly increase the thoroughout of computation, in reality, we are bounded by memory, I/O, and network.

\subsubsection{R}

Several parallel programming packages in R include parallel, caret, multidplyr, and doParallel \cite{jones2017quick, rubia2017multicore}.
Different packages are useful for different situations like the parallel::mclapply function. It is useful for repetitive 
independent tasks that can be done in parallel.  

\subsubsection{GRASS}

Parallel programming in GRASS is interesting because GRASS is not a monolithic application but consists of over 300 
autonomous modules with their own memory management and error handling. An article I read used a shared memory 
parallelization tool (OpenMP) in GRASS \cite{hofierka2017parallel}. In the Literature Review, I will discuss this article 
more but from what I understand there is more freedom for parallel programming in GRASS than R. R feels very restrictive 
in the packages and functions but it is easier to implement.  

\subsection{Difficulties/Questions}
I am trying to learn the concepts and architectures because I believe we need to find the best fit for the situation. 
So, I am trying to learn about the best possible solution for a parallel k-means clustering for large datasets.
\begin{itemize}
    \item Should we use a distributed or shared memory?
    \item How can we separate a k-means clustering into independent tasks?
    \item What are the bounds for our algorithm?
    \item How should we analyze the optimal amount of threads?
\end{itemize}
I am hoping to start practicing and coding parallel functions to deepen my learning next week because it is difficult for
me to fully grasp the concepts without trying it myself.

\section{Literature Review}

The article I focused my time on throughly reading was "Parallelization of interpolation, solar radiation and water 
flow simulation modules in GRASS GIS using OpenMP". The study explored the implementation of shared memory parallelism
for three computationally intesive modules: v.surf.rst, r.sun, and r.sim.water. The article throughly went through the
thought process and steps that came into the parallelization of their code and I found this extremely useful because I
am hoping to implement the same steps when I arrive at that step in my process. The article was also very through on 
describing the terminology easily for those learning parallel programming and providing sources of other sources. \cite{hofierka2017parallel}

The assigned article for this week had me thinking of the location of my study area. In the article, they focused on California
because the current and future trends in climate will impact the agriculture and economy of California. It is a very diverse
landscape and permanent crops will be hit hard because they commonly grow for more than 25 years which make them vulnerable to
impacts in climate. I am currently not strict on the location I want to focus on but, California is high on my list. \cite{pathak2018climate}

% this info creates the bibliography
% YOU WILL NEED TO CHANGE THIS PATH TO THE LOCATION OF THE BIB file
\bibliography{./agclimate.bib}
\bibliographystyle{plain}


\end{document}
