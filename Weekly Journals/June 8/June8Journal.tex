\documentclass[a4paper,10pt]{article}
\usepackage[utf8]{inputenc}

% increase margins
\usepackage{fullpage}
\usepackage[left=1in,top=1in,right=1in,bottom=1in,headheight=3ex,headsep=3ex]{geometry}

% this puts two lines between paragraphs and no indent
\usepackage[parfill]{parskip}

% set up colors
\usepackage{array, xcolor}
\usepackage{color,hyperref}
\usepackage{graphicx}

\definecolor{torontoblue}{HTML}{00204E}
\definecolor{linkblue}{HTML}{0000FF}

% define hyperlink style
\hypersetup{colorlinks,breaklinks,
            linkcolor=linkblue,urlcolor=linkblue,
            anchorcolor=linkblue,citecolor=linkblue}


%opening
\title{Weekly Journal}
\author{Leila Uy}



\begin{document}

\maketitle

\section{Work Update}
This week, I played around with EC2 and S3 AWS, the code base, and read some articles. 
\cite{carruthers2003invasive, wang2019potential, peng2017economic}

\subsection{AWS}
The majority of my time was spent getting familiar with EC2 and S3. I installed R in the instance and test ran some  
simple R script through the terminal. I wanted to test R script that works on some sample data by connecting the EC2 
instance and S3 bucket, but I got lost through the process. Although I made the bucket public, I kept getting a 403 
Access Denied Error and Unable to locate credentials. There is so much documentation for AWS that I found myself 
drowning in it with over 30 open tabs of documentation and Stack Overflow, but I want to try again this week as 
a group.

\subsection{Code Base}
I was always told by a friend that if you want to learn code, you should always type it yourself to truly understand 
the importance of each line. So after some frustration with AWS, I moved to the code base so I could start implementing 
the non-parallel version of the K-means clustering code. I started an R Notebook and went through the code and checking 
the data through the process using North Carolina as my study area. I changed the CRS of the boundary to WGS84 in 
QGIS but I forgot to remove county boundaries so I used gUnaryUnion in R, which could be the reason why mask is not 
working. I will attach my R Notebook when I submit this journal.

\section{Literature Review}
This week I mostly focused my articles on the effects of climate change on agriculture. An article by Peng et al. 
discusses the importance of additional climatic variables (other than temperature and precipitation) on studying 
the impacts of climate change on agriculture. They researched several variables like humidity, wind speed, sunshine 
duration, and evaporation from agricultural data (rice, wheat, corn) from 1980 to 2010 in China. They found that 
because the relationship between tempearture and other climatic variables are not static, the correlation varies 
over time. For example although humidity tends to increase during climate change, humidity decreases during the 
warmest months because of the increase in extremely high temperatures. In addition, wind speeds are expected to 
increase because of climate change. Therefore, the omission of both humidity and wind speeds will result in the 
underprediction of the cost of climate change. \cite{peng2017economic}

Two other articles I read were focused around climate change, agriculture, and invasive plant species (IPS). The 
potential of invasive plant expansion is expected to increase due to changing ecoregions resulting from climate 
change. They used WorldClim data and found that the most important climatic suitability variable for IPS was 
annual mean temperature and temperature seasonality. Places that predict to have the greatest risk of expansion 
is Northern Europe, the UK, South America, North America, southwest China, and New Zealand.  \cite{wang2019potential} 
An article published in 2003 talks about research of invasive species in agriculture. It foound that many exotic insect, weed, and 
pathogen species are responsible for a large portion of the \$130 billion losses estimated to be caused by pests 
each year. Although the United States's ARS scientists are taking great lengths to research invasive species, the 
scope is too big. The number of exotic organisms plus the many that have not been recognized are making it difficult 
for researchers to plan accordingly. \cite{carruthers2003invasive}



% this info creates the bibliography
% YOU WILL NEED TO CHANGE THIS PATH TO THE LOCATION OF THE BIB file
\bibliography{./agclimate.bib}
\bibliographystyle{plain}


\end{document}
