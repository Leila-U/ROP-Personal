\documentclass[a4paper,10pt]{article}
\usepackage[utf8]{inputenc}

% increase margins
\usepackage{fullpage}
\usepackage[left=1in,top=1in,right=1in,bottom=1in,headheight=3ex,headsep=3ex]{geometry}

% this puts two lines between paragraphs and no indent
\usepackage[parfill]{parskip}

% set up colors
\usepackage{array, xcolor}
\usepackage{color,hyperref}
\definecolor{torontoblue}{HTML}{00204E}
\definecolor{linkblue}{HTML}{0000FF}

% define hyperlink style
\hypersetup{colorlinks,breaklinks,
            linkcolor=linkblue,urlcolor=linkblue,
            anchorcolor=linkblue,citecolor=linkblue}


%opening
\title{Literature Review Draft}
\author{Leila Uy}



\begin{document}

\maketitle

%this is commented out, no need for abstract in  your weekly assignment
% \begin{abstract}
% 
% \end{abstract}

\section{Introduction}

There is a growing part of research creating trajectories and looking at the effects of climate change using past and present data. The Intergovernmental Panel on Climate Change (IPCC) has developed four storylines and 40 scenarios through their emissions scenario report \cite{IPCC2000emissions}.  An assessment by the IPCC measured the combined impacts of climate change and acidification of agricultural crops in Asia; the assessment predicts that agricultural productivity in the Indian subcontinent will reduce due to projected temperature and precipitation increases. In contrast, the assessment expects that regional climate change will lead to increased agricultural output in China, although high levels of acidic deposition may offset any possible benefits. This natural relationship between climatic variables and vegetation growth makes the agricultural sector vulnerable to any changes, directly or indirectly, witnessed by future climate change scenarios. 

The correlation between temperature and additional climatic variables fluctuates over time, and this can make predicting the effects of climate change complicated. For example, the relationship between temperature and humidity varies over time due to the distribution, intensity, and frequency of precipitation \cite{peng2017economic}. These variables, in turn, alter the ecological suitability of the area for species, including agricultural crops and invasive plant species (IPS). Wang et al. \cite{wang2019potential} suggests IPS  may result in landscape homogeneity at ecoregional scales as the invasive species will compete with native plant species. They estimate that, in the United States, many exotic insects, weeds, and pathogens were responsible for a large portion of the \$130 billion losses each year in the agricultural sector \cite{carruthers2003invasive}. To predict the impacts of climate change on agriculture, there must be a way to explore the shifts in suitability for different species' ranges. 

Geographical ecoregions consist of regions with similar ecological and environmental conditions. They are important to consider when estimating the extent of suitable habitat for particular species, including agricultural crops and competing IPS. While several academic papers have identified and classified ecoregions using different methods, there is no established classification system. A 2001 study by Olson et al. \cite{olson2001terrestrial} used human experts to create a detailed terrestrial ecoregion classification that was better suited to identify areas of outstanding biodiversity. Another 2008 study by Ellis et al. \cite{ellis2008putting} introduced eighteen anthropogenic biomes because previous global patterns of biodiversity either ignore or simplify the direct interaction between humans and ecosystems. A 2014 study that included a collaboration between the Association of American Geographers (AAG), the United States Geological Survey (USGS), Esri, and the Group on Earth Observations (GEO) \cite{globaleco2014a}, approached classification by stratifying the Earth into physically distinct areas with their associated land covers. Traditional methods of classification often delineate ecoregions by hand. Consequently, results are often difficult to replicate, and introducing more variables and large data sets takes a long time.


As we grow into a new age with more accessible and higher resolution spatial data sets, it is important to create an accurate, quick, and flexible classification system for determining ecoregions. K-means clustering was independently discovered in different scientific fields as far back as the 1950s and remained popular due to its simple implementation, which has introduced several extensions many years after its introduction \cite{cuomo2019a,dunn1973a,jain2010data,pelleg1999accelerating,scholkopf1998nonlinear,tang2017parallel}. A recent implementation of this multivariate statistical clustering algorithm is the integration of parallel programming and computational optimization to speed up the process. Hargrove and Hoffman \cite{hargrove1999using} developed a parallel multivariate geographic clustering algorithm in C using the Message Passing Interface (MPI). Rather than relying on expertise for classification, their method places individual raster cells in a data space where each dimension represents an environmental characteristic.  For example, when analysing precipitation, mean temperature, and wind speeds, each raster cell would have a coordinate based on the three variables (e.g., $u_{1} = (0, 0.5, 23)$). Kumar et al. \cite{kumar2011parallel} later optimized this implementation using the triangular inequality to remove unnecessary calculations.

The focus of the work reported here is the design and implementation of a parallel K-means clustering algorithm in R that is easily adjusted for n variables, based on previous literature in parallel clustering and ecoregions. Utilizing the computational powers of AWS EC2 machines and large historical and future climate data provided by WorldClim, we test the accuracy, speed, and flexibility of the algorithm as a classification system for predicting shifts in ecoregions resulting from climate change.


% this info creates the bibliography
% YOU WILL NEED TO CHANGE THIS PATH TO THE LOCATION OF THE BIB file
\bibliography{./agclimate.bib}
\bibliographystyle{plain}


\end{document}
