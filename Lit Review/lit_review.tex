\documentclass[a4paper,10pt]{article}
\usepackage[utf8]{inputenc}

% increase margins
\usepackage{fullpage}
\usepackage[left=1in,top=1in,right=1in,bottom=1in,headheight=3ex,headsep=3ex]{geometry}

% this puts two lines between paragraphs and no indent
\usepackage[parfill]{parskip}

% set up colors
\usepackage{array, xcolor}
\usepackage{color,hyperref}
\definecolor{torontoblue}{HTML}{00204E}
\definecolor{linkblue}{HTML}{0000FF}

% define hyperlink style
\hypersetup{colorlinks,breaklinks,
            linkcolor=linkblue,urlcolor=linkblue,
            anchorcolor=linkblue,citecolor=linkblue}


%opening
\title{Weekly Journal}
\author{Leila Uy}



\begin{document}

\maketitle

%this is commented out, no need for abstract in  your weekly assignment
% \begin{abstract}
% 
% \end{abstract}

\section{Literature Review}
There is a growing part of research creating trajectories and looking at the effects of climate change 
using past and present data. An IPCC emissions scenario report created four storylines and 40 scenarios.  
One assessment looked at the combined impacts of climate change and acidification of agricultural crops 
in Asia. They found that the Indian subcontinent would have lower agricultural productivity due to 
projected temperature and precipitation. In contrast, regional climate change would benefit China, 
increasing their agricultural output, but high levels of acidic deposition would offset any possible 
beneficial impacts \cite{IPCC2000emissions}. Due to this natural relationship between climatic variables and 
vegetation growth, the agricultural sector is extremely vulnerable to any changes witnessed by future climate 
change scenarios whether directly or indirectly. 

Ideally, if the correlation between temperature and additional climatic variables were fixed over time, predicting 
the effects of climate change would be simple, but for example, the relationship between temperature and humidity 
varies over time due to the distribution, intensity, and frequency of precipitation \cite{peng2017economic}. These 
variables, in turn, rearrange the ecological suitability of the area for species including agricultural crops and 
invasive plant species (IPS). Studies suggest that IPS have a large potential to result in landscape homogeneity at 
ecoregional scales and compete with native plant species \cite{wang2019potential}. It is predicted that many exotic 
insect, weed, and pathogen species were responsible for a large portion of the \$130 billion losses estimated to be 
caused by pests each year in the United States' agricultural sector \cite{carruthers2003invasive}. In order to 
predict the impacts of climate change on agriculture, there must be a way to explore the shifts in suitability for 
different species' ranges. 

Geographical ecoregions are the identification of different regions that are similar in ecological and environmental 
conditions. They are important when estimating the extent of suitable habitat for particular species including 
agricultural crops and competing IPS. A simple search through academic papers on ecoregions will give you over a 
hundred results because there is no established classification system. A 2001 study by Olson et al. \cite{olson2001terrestrial} 
used human experts to create a detailed terrestrial ecoregion classification that was better suited to identify areas 
of outstanding biodiversity. Another 2008 study by Ellis et al. \cite{ellis2008putting} introduced eighteen 
anthropogenic biomes because previous global patterns of biodiversity either ignore or simplify the direct 
interaction between humans and ecosystems. A 2014 study that included a collaboration between Esri, USGS, AAG, and 
GEO \cite{globaleco2014a}, approached classification by stratifying the Earth into physically distinct areas with 
their associated land covers. Traditional methods of classification often delineated ecoregions by hand and as a 
consequence, results are often difficult to replicate, and introducing more variables and large data sets would 
take a long time.

As we grow into a new age with more accessible and higher resolution spatial data sets, it is important to create 
an accurate, quick, and flexible classification system for determining ecoregions. K-means clustering was 
independently discovered in different scientific fields as far back as the 1950s and remained popular to this 
day due to its simple implementation which has introduced several extensions many years after its introduction 
\cite{cuomo2019a,dunn1973a,jain2010data,pelleg1999accelerating,scholkopf1998nonlinear,tang2017parallel}. 
A recent implementation of this multivariate statistical clustering algorithm is the integration of parallel 
programming and computational optimization to speed up the process. Hargrove and Hoffman \cite{hargrove1999using} 
developed a parallel multivariate geographic clustering algorithm in C using the Message Passing Interface (MPI). 
Rather than relying on expertise for classification, individual raster cells were placed in a data space where 
each dimension represented an environmental characteristic. So, if you were to look at precipitation, mean 
temperature, and wind speeds, each individual raster cell would have a coordinate based on the three variables 
(e.g. $u_{1} = (0, 1, 0)$). This implementation was later optimized by Kumar et al. \cite{kumar2011parallel} by using the 
triangular inequality to remove unnecessary calculations.

The focus of the work reported here is the design and implementation of a parallel K-means clustering algorithm in 
R that is easily adjusted for n variables, based on previous literature in parallel clustering and ecoregions. 
Utilizing the computational powers of AWS EC2 machines and large historical and future climate data provided by 
WorldClim, we are testing the accuracy, speed, and flexibility of the algorithm as a classification system for 
predicting shifts in ecoregions as a result of climate change.


% this info creates the bibliography
% YOU WILL NEED TO CHANGE THIS PATH TO THE LOCATION OF THE BIB file
\bibliography{./agclimate.bib}
\bibliographystyle{plain}


\end{document}
